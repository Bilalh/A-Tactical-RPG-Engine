%% Testing
%!TEX root = ../Project.tex
\section{Testing Summary}

Unit testing\footnote{Using the JUnit library.} was used for the testing on the model. These can be rerun in the Eclipse IDE or from the command line which produces a webpage of the results.

\begin{lstlisting}[caption=Commend to make a webpage of the result of the unit testing]
	ant tests
\end{lstlisting}

In addition I conducted user testing using a suverty as described in section \ref{sub:results_of_user_testing}. 

Before the formal user testing, I make the froling changes in responses to comments. 

\begin{itemize}
	\item It was hard to see which unit was selected. This was fixed by  displaying `Current' in selected unit's info. The info window of the selected unit was also lightened to to make it more obvious.
	
	\item It was hard to see which are my units. This was fixed by displaying the player's unit's info in green and the enemy's unit's info in red. 
	
	\item Some users could not figure out the key bindings of the game. This was fixed by displaying a list of all key binding at the start of the game.
\end{itemize}
