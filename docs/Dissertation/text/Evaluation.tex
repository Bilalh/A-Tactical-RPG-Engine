%% Evaluation
%!TEX root = ../Project.tex
\section{Evaluation and Critical Appraisal}

\subsection{Objectives}
\label{sub:eobjectives}
The objectives as listed in section \ref{sub:objectives}  are extavise, ranging from primary objectives that required a basic engine capable of creating a Tactical RPG to tertiary objectives which required a creation of a editor for the engine. 

Since all the primary and secondly objectives were completed as well as nearly all the tertiary objectives I consider  the project a success, I will now discuss the objectives in details.

\subsubsection{Scripting \& Custom Events}
A tertiary requirements was to implement custom events though the use of scripting.  This would allow the user to specify events such as making the player win if some enemies unit has less then 50\% Hit Points. 

To implement this I had planed to use the Java Scripting API\cite{javas}. The would allow embedding a dynamic scripting language.  The main benefit of this, is that do not have to be compiled which means it have the same hundles with class linking as my Java extension method.   The language chosen for embedding was Javascript  as it is build into the JDK, so the user does not have install anything extra and of all the scripting language Javascript  is more likely to have used before by the average user, these desctions are desction in more detail in section \ref{sec:Scripting}. 

Some such as ref  advatage forgoing scripting completing. Pointing out that scripting is a lazy out of let user customiseablity since it can be quite for user to create events.  Using their advice of creating visual editors for creating events instead I allows the user to specify common events from the editor, with the option of link they own classes as a last reosct. 

Algoth the scripting was not implementation due to time constraints, note that it was  tertiary objectives and all the example can be still be specified. In perticual win condcustion can be specified (section ref) and common winning condcustion can be selected from the editor.  The engine support shows dialog at abtaution point in the game which could be used to support the requirement ``Showing some part of the story when a player’s character reaches a specified tile''.

The engine supports the user using their own classes though the extension mechanism in Appendix \ref{sub:data_format_extension_mechanism}, which allows a slightly more limited form of scripting but with the benefit of being type safe but with the drawback of being  more complex.
% ref game progamming gems

% morrowind 
% wow 
% never winter nights 
% 

% ref example reportt
\subsection{Results of User Testing}
In total six participants took part in the user testing, more people were interested but could not due to time constraints.  See Appendix \ref{sec:questionnaire} for the Questionnaire.

\subsection{Analysis of Responses}
The participants can be grouped into three groups, those who have played a TRPG before, those who have not and those who don't play games at all.
\paragraph{Feedback from  all groups\\}
Both groups thought the editor supported all the features they would like if they created a TRPG.  Most of the participants answered that they would like to use the editor again.  Interestingly, all users who were familiar with TRPG used the keyboard exclusively, this I attribute to participants being used to the input mechanism since most TRPG are on consoles\footnote{such as Playstation 2}. Those who had no experience playing TRPG unexpectedly used both the keyboard and mouse at the same time (generally selecting the units with the mouse then using the keyboard for performing any actions).

\paragraph{Feedback from those who had played a TRPG before\\}
Participants that had experience playing a TRPG before found the engine very intuitive. They especially liked how everything was customisable.  The visual map editor also gained significant praise for it's easy of use and it's range of features.  Other features that participants liked were how a project could be exported as a standalone Mac OS X application with virtually no effort. 

When playing the pre-created game, all of the participants were able to play, and even win, the game!. Some of the participants were able to figure out all of the out the controls with out even reading the help screen. 

\paragraph{Feedback from those who had not played a TRPG  before\\} 
In contrast, participants who never played a TRPG before felt overwhelmed with the available options, although they figured how to use it quite quickly.  They particularly liked how robust the editor was. The main features that were requested were  an overview of each feature in the editor and how they interacted.  The participants commented on the pleasing visual appearance of the editor.

When playing the pre-created game, some of the participants were initially confused by the interface. All participant were able to finish the game even if they could not win.  The participant unexpectedly kept on using the enter key to perform actions on the unit even though the help panel stated otherwise. In response to this feedback, I added the enter key as an alias of the action key(default `x').

\subsubsection{System usability scale}
A System Usability Scale (SUS) was used \cite{SUS}. This works by giving each even numbered questions a score of (5 - \emph{value}) and odd numbered questions a score of (\emph{value}-1). Questions that contributed to a high score showed that the system is usable. 

Based on this schema, the maximum positive contribution is four. To get the overall usability score, the sum of the questions contributions is multiplied by 2.5. 

\subsubsection{Results}
The usability result was 61. This is a very good result since a score of 50 means that the system is useable. 

\subsubsection{Analysis}
Since no individual question scored below two, the system has most of the features the users wanted.