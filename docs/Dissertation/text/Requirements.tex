%% Requirements Specification
%!TEX root = ../Project.tex
\section{Requirements Specification}

\subsection{Project Scope}
\label{sub:project_scope}

The aim of the project is to enable the user to create a highly customisable Tactical RPG.  There are three main parts to the project: the \emph{engine}, the \emph{GUI} and the \emph{editor}.

The engine will contain all the logic of the game including handling the game progression as well as the battle system. The GUI will be an isometric view of the game (see Section~\ref{sub:tilemap}). 

The editor will allow the user to specify the input to the engine. This includes visual map making as well as specifying all the attributes of the units and weapons in addition to the winning conditions.  The editor will also allow the user to export the game as a standalone application.

\subsection{Requirements Overview}
\label{sub:overview}
A complete listing of requirements is in section~\ref{sub:objectives}. The main requirements are to create an engine which allows a high degree of customisability, an isometric view and the ability to export the game as a standalone application.  

\subsection{Overview Description}
\subsubsection{Product Perspective}

\subsubsection{Product functions}

\subsubsection{User characteristics}
The system is for anyone that would like to create a TRPG.  They need not have any experience in creating TRPG before, or even played one before since the game can be created without \texttt{any} programming. 

For more advanced users, they can further customise the created game using their own code. This allows the user to completely change most aspects of the game. This could be used for example to make unique abilities or battle systems.
%TODO talk about java scripting       


\subsubsection{Constraints, assumptions and dependencies}
The system should be portable i.e it should work on most operating systems.  To achieve this I chose to use Java  since it would work on any system that has the Java runtime environment installed on it. 

%CS4099 Dissetation
%!TEX root = ../Project.tex
\section{Objectives}

\begin{figure}[h!]
	\caption*{Key}
	\begin{center}
		\begin{tabular}{rl}
		\Checkmark & Finished\\
		\XSolidBrush & Not Started\\
		– $\bullet$ $\star$ & In progress\\
		\end{tabular}
	\end{center}
\end{figure}

\label{Objectives}
\subsection{Primary}
\label{primary}
The main goal of the primary objectives is allow the user to create a complex Tactical RPG, with limited customisability.  
\begin{itemize}
\item To develop an engine that takes:
\begin{itemize}

 \item The definition of character attributes and a combat system.
	\item The definition of a world broken up into the smaller environments.
	\begin{itemize}
		\item The rules of the game.
		\tick The kinds of enemies.
	\end{itemize}
	
	\tick The definition of simple story as a wrapper for the whole game, from the start to the conclusion of the game
	\begin{itemize}
		\tick Which is told between the movement between different environments.
	\end{itemize}
	                        
\end{itemize}
and create a playable tactic RPG.

\item To include in the engine support for the following:

\begin{itemize}
	\tick \texttt{units} with a fixed set of associated attributes such as:
	\begin{itemize}
		\tick Hit-points (which represent the health of the unit).
		\tick Strength.
		\tick Defence.
		\tick Move (The number of tiles the unit can move each turn).
	\end{itemize}
	
	\item \texttt{battles} which take place on grid and include:
	\begin{itemize}
		\tick  A set number of \texttt{units} for each player.
		\tick  A Winning \texttt{condition}, which is defeat all of the other player's units.
		\tick  Battles are \texttt{turn based} meaning only one unit performs at one time.   
		\item  A combat system. \marginpar{\textbf{Nearly}}
	\end{itemize}
	
	\item A combat system that includes
		\begin{itemize}
			\tick \texttt{combat} between adjacent units.
			\begin{itemize}
				\tick When the unit hit-points are reduced to zero they are \texttt{defeated} and are removed from the map
			\item A set of rules that govern the combat.
			\end{itemize}
			
		\end{itemize}
	
	\item A predefined set of behaviours for how the non-player characters should behave.
	\begin{itemize}
		\tick Including pathfinding.
	\end{itemize}
	
	\tick A isometric graphical representation of the game.
	\begin{itemize}
		\tick Which is show the grid with all the units.
		\tick Allow the user to move their units and see the opponents moves.
		\tick Allows the user to attack the opponents units.
		\tick Which allows the user to see a unit status (e.g current hit points).
		\tick Text will be to describe the more complex actions such magic.
	\end{itemize}
\end{itemize}
\end{itemize}

\subsection{Secondary}
\label{secondary}

The main goal of the secondary objectives is allow the user more customisability. 
\begin{itemize}
	\tick Tiles have \texttt{height}, where units can only move to tiles of a smilier height.
	\item Tiles that are not passable such as sea, lava, etc.
	
	\tick Tiles have different movement costs associated with them.
	
	\tick A combat system that includes 
	\begin{itemize}
		\tick \texttt{combat} between non-adjacent units,
	\end{itemize}
	
	\tick Players have items such as weapons that affect the result of combat between units. 
	\begin{itemize}
		\tick Including long distance weapons for the player and AI.
	\end{itemize}
	
	\cross Direction and height of the character's tile affects attack.
	
	\tick Sound effects.
	\tick Music.
	
	\cross Saving and loading games.
	
	\item Allow the user to specify some of behaviour of non-player characters
	\begin{itemize}
		\cross An example: always attack a certain kind of unit or always attack the unit with the least Hit Points.
	\end{itemize}
	
	\tick A graphical view to allow user specify input to the engine.
\end{itemize}

\subsection{Tertiary} 
\label{tertiary}
The goal of the Tertiary objectives are provide the user with more customisability and to provide a GUI for customise aspects of the engine. 

\begin{itemize}
	
	\tick A combat system that includes 
	\begin{itemize}
		\tick Support for \texttt{skills} which can effect multiple units.
		\tick Including weapons that can attack multiple units at the same time. 
	\end{itemize}
	
	\tick Animations for units and movement.
	
	\item A graphical editor for  creating and specifying the input to the engine which allows:
	\begin{itemize}
		\tick   Creating and editing maps.
		\begin{itemize}
			\tick which also allows placement of enemy units.
		\end{itemize}
		
		\item   specifying the order of the maps.
		\tick   making animations.
		\tick   making items such as weapons.
		\tick   making skills. 
		\tick   making units.
		\tick   specifying the story, at the start and end of a battle.
		\tick   specifying the music and sound effects played on each map.
		\tick   specifying the condition to win a map such as:
		 \begin{itemize}
		 	\tick Defeating all the opponent's units.
		 	\tick Defeating a specific unit.
		 \end{itemize}
	\end{itemize}
	
	\cross Custom events
	\begin{itemize}
		\item Attached to units or titles, could be used for:
		\begin{itemize}
			\cross Making the player win if some enemies unit has less then 50\% Hit Points.
			\cross Damaging a character if step on a specified.
			\cross Showing some part of the story when a player's character reach a specified tile.
		\end{itemize}
	\end{itemize}
	
\end{itemize}

\subsection{Specific Requirements}
The system should allow the user to export the created game with no additional dependencies apart from the Java runtime environment. 

\subsubsection{Security Requirements}
Although their are no security requirements in the objectives, they could nevertheless  be considered in the future. Xml is used as the data format for the created games. This aids maintainability since xml is human readable and there are many tools to manipulate the files. The disadvantage of this is that the end users of the created game can also access the data, hence they could edit it or steal the resources of the game. A solution to this problem would be to encrypt the data files so they are not editable by the end users.    

\subsubsection{User Interface Requirements}
The GUI should provide a isometric view of the game as shown in \ref{fig:TRPG}. The GUI should visually display to the user  which action the opponent performs. The GUI should give visual feedback for any actions the user makes.

% \subsection{Technology Architecture}


