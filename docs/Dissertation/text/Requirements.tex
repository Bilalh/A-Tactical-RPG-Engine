% Requirements Specification
%!TEX root = ../Project.tex
\section{Requirements Specification}

\subsection{Project Scope}
\label{sub:project_scope}

The aim of the project is to allow the user a highly customisable Tactical RPG.  There are three main parts to the project the \emph{engine}, the \emph{GUI} and the \emph{editor}.

The engine will contains all the logic of the game including the progression as well as the battle system. The GUI, will be an isometric view of the game (see Section~\ref{sub:tilemap}). 

The editor will allows the user to customise the input to engine. This includes visual map making as well as specify all the attributes of the units and weapons.  The editor also allows the user to export the game as a standalone application.

\subsection{Requirements Overview}
\label{sub:overview}
A complete listing of requirements is in section~\ref{objectives}. The main requirements are to create a engine allows a high degree of customisability, a isometric view and exporting the game as a standalone application.  

\subsection{Overview Description}
\subsubsection{Product Perspective}

\subsubsection{Product functions}

\subsubsection{User characteristics}
The system is for anyone would like to create a TRPG.  They need not have any experience in creating TRPG before, or each played one before since the a game can be created without \texttt{any} programming. 

For more advance users, they of course further customise the created game using their own code. This allows the user to completely change most aspects of the game. This could be used for example to make unique abilities or battle system.
%TODO talk about java scripting       


\subsubsection{Constraints, assumptions and dependencies}
The system should be portable i.e it should works on most operating systems.  To achieve this I used Java achieve since it would work on any system that has the java virtual machine installed on it. 

%!TEX root = ../Project.tex
\section{Objectives}
\label{Objectives}
\subsection{Primary}
\label{primary}
The main goal of the primary objectives is allow the user to create a complex Tactical RPG, with limited customisability.  
\begin{itemize}
\item To develop an engine that takes:
\begin{itemize}

 \item The definition of character attributes and a combat system.
	\item The definition of a world broken up into the smaller environments.
	\begin{itemize}
		\item The rules of the game.
		\item The kinds of enemies.
	\end{itemize}
	
	\item The definition of simple story as a wrapper for the whole game, from the start to the conclusion of the game
	\begin{itemize}
		\item Which is told between the movement between different environments.
	\end{itemize}
	                        
	\item The set of selected character attributes.
	
\end{itemize}
and create a playable tactic RPG.

\item To include in the engine support for the following:
\begin{itemize}
	\item \texttt{units} with a fixed set of associated attributes such as:
	\begin{itemize}
		\item Hit-points (which represent the health of the unit).
		\item Strength.
		\item Defence.
		\item Move (The number of tiles the unit can move each turn).
	\end{itemize}
	
	\item \texttt{battles} which take place on grid and include:
	\begin{itemize}
		\item A set number of \texttt{units} for each player.
		\item A Winning \texttt{condition} such as defeat all of the other players units.
		\item Battles are \texttt{turn based} meaning that each player moves all their units (once) before the next player turn.   
		\item A combat system.
	\end{itemize}
	
	\item A combat system that includes
		\begin{itemize}
			\item \texttt{combat} between adjacent units.
			\begin{itemize}
				\item When the unit hit-points are reduced to zero they are \texttt{defeated} and are removed from the map
			\item A set of rules that govern the combat.
			\end{itemize}
			
		\end{itemize}
	
	\item A predefined set of behaviours for how the non-player characters should behave.
	\begin{itemize}
		\item Including pathfinding.
	\end{itemize}
	
	\item A simple graphical representation of of the game.
	\begin{itemize}
		\item Which is show the grid with all the units.
		\item Allow the user to move their units and see the opponents moves.
		\item Allows the user to attack the opponents units.
		\item Text will be to describe the more complex actions such magic.
	\end{itemize}
\end{itemize}
\end{itemize}

\subsection{Secondary}
\label{secondary}
The main goal of the secondary objectives is allow the user more customisability. 
\begin{itemize}
	\item Tile \texttt{height}, where units can only move to tiles of a smilier height.
	
	\item Tiles that are not passable such as sea, lava, etc.
	
	\item Tiles have different movement costs associated with them.
	
	\item Isometric graphics view of the game.
	
	\item Long distance weapons\slash magic for player and AI.
	
	\item Direction and height of the character's tile affects attack.
	
	\item Sound effects.
	
	\item Music.
	
	\item Saving and loading games.
	
	\item Allow the user to specify some of behaviour of non-player characters
	\begin{itemize}
		\item Through the use of scripting.
		\item An example: always attack a certain kind of unit or always attack the unit with the least Hit Points.
	\end{itemize}
	
	\item A graphical view to allow user specify the input to the engine.
\end{itemize}

\subsection{Tertiary} 
\label{tertiary}
The goal of the Tertiary objectives are provide the user with more customisability and to provide a GUI for simple scripts. 

\begin{itemize}
	\item Custom events
	\begin{itemize}
		\item Attached to units or titles, could be used for:
		\begin{itemize}
			\item Making the player win if some enemies unit has less then 50\% Hit Points.
			
			\item Damaging a character if step on a specified.
			
			\item Showing some part of the story when a player's character reach a specified tile.
		\end{itemize}
	\end{itemize}
	
	\item A graphical editor for making custom maps, events and specifying the input to the engine.
	\begin{itemize}
		\item The gui would also be able to create the scripts for simple event such as `Defeat the leader' as a winning condition.
	\end{itemize}
	\item Animations for units and movement.
\end{itemize}

\subsection{Specific Requirements}
The system should allow the user to export the created game with no addicanl dependencies apart from the java runtime evnvment. 

\subsubsection{Security Requirements}
Although their are no security requirements in the objective, they should nevertheless  be considered in the future. Xml is used as the data format for the created games. This aids maintainability since xml is human readable. The disadvantage of this is that user of the created game can also access the data, hence could edit it or steal the resources of the game. A solution to this problem would be to encrypt the data files so they are not editable by the end users.    

\subsubsection{User Interface Requirements}
The GUI should provide a isometric view of the game as shown in \ref{fig:TRPG}. The GUI should the user to visually see which action the opperent performs. The GUI should give visual feedback for any actions the user makes

% \subsection{Technology Architecture}


