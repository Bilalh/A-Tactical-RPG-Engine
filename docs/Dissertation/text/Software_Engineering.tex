% Software Engineering
%!TEX root = ../Project.tex
\section{Software Engineering Process}

\subsection{Methodologies Used}
\label{sub:methodologies_used}

I chose to use a iterative spiral development model for the project. This allowed me to focus on specific parts of the system before moving onto the next component. 

Prototypes were extensively used especially in the GUI when choosing how to render the map. %TODO REF


\subsubsection{Test Driven Development}
Test-driven development (TDD) was utilised in the project. This also helped with verification of requirements as tests assert whether the code matched the minimum requirements. The \texttt{JUnit} library\footnote{JUnit 4.8.2, see \url{www.junit.org/} for details } was used to write the unit tests.  

The main stages in TDD are\cite{murphytest},\cite{desai2008survey}:
\begin{enumerate}[noitemsep ]
	% \item Write a Test.
	% \item Write the code to pass the test.
	% \item Run all tests.
	% \item Refractor
	\item Before the code is written, unit tests to test the functionality is created. These will initially fail.
   \item Code is written to pass the test and no more.
   \item If more functionality is required, first the test is written and then the code to pass it.
   \item Changes to the previously written code must pass all previous created tests.
   \item The code is refracted 
\end{enumerate}

The major benefits of TDD are that system will be well tested.  A additional benefit is that  it  prevents new features from introducing bugs. Combined with version control as discussed in the next section, it makes it very easy to find bugs since the unit tests can be used to find out \emph{when} the code stop working as well as to find out \emph{which} piece of code was the root cause.

This method of development was perfectly suited to implementing the algorithms in the engine (such as unit movement) because the expected output was known beforehand.  Since all components of the model were programmed to an interface, it allows the use of mock objects\ref{sec:mockobjects}.   

%To further ease testing of the algorithms I used mock object  

However TDD has few drawbacks such as the difficulty of realising all possible test scenario, which becomes apparent when testing the GUI and the editor. To test these aspects of the system I played multiple created game from start to finish with the goal of finding any lingering bugs. In addition I did user surveys as well as usability studies to find any unexpected defects in the user interface. (results in ) 
% TODO ref

\subsection{Mock Objects}
\label{sec:mockobjects}
Mock objects are used predominantly in very large software development projects to aid in testing. The objective is to create objects which simulate only the essential behaviour of the object required. 
Mock objects abstract  the detailed functionality of the implementation away and  focuses only on what is required for the test.

\subsection{Version Control}
\label{sub:version_control}
Version control keeps track of all changes to a project. It keeps the history of changes and  helps to detect when a bug was first introduced, hence cause of the bug. In particular I chose to use \texttt{git}, which is distributed version control system. It differs from traditional client server system such as Subversion in that each user has a 
complete copy of the repository. 

Distributed version control system have the advantage of allowing changes to be committed locally, even without an internet connection.  This was particularly useful for this project since it allowed experimenting with various features before choosing the features to integrate into the system.
  

%TODO ref