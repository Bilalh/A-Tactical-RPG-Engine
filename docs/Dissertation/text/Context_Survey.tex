%%Context Survey
%!TEX root = ../Project.tex

\section{Context Survey} 
\label{Context_Survey}

\subsection{Evolution of Tactical RPGs }
\label{sub:evolution_of_tactical_rpgs_}

Classic Tactical RPGs(TRPGs)  were heavy influenced traditional Strategy games in particular chess.  The earliest game that shares the concepts of a TRPG is ``Bokosuka Wars''(1983) had many of features now common in TRPGs include grid based combat with a number of units\cite{BokosukaWars}\footnote{Tactical RPG also are called Simulation Strategy RPG(SRPG),  most commonly in Japan.}. 

Although there many western games that had smilier gameplay at the time like ``Ultima III: Exodus''(1985), which is a traditional RPG with TRPG like battles, they quickly diverged and soon featured real time combat, or reverted to a simple combat system traditional RPG like ``Final Fantasy''.  This tread lead to most TRPG until recently be made in Japan since until recently western publishers  thought TRPGs were to niche to be profitable \cite{notrpg}. 

I will briefly describe  the games that made a significant contributions to the genre, these will be used evaluating the capabilities of the created engine.

\begin{itemize}
	
	\item \emph{Fire Emblem}(1990), popularised the genre. The game induced RPG elements such as weapons, skills that became standard in later game. It also was also one of the first TRPG to have a non-trivial plot. 

	\item \emph{Tactics Ogre}(1994) the game which started the to have the name tactics in the game  which gave rise to the Tactical RPG name\footnotemark[\value{footnote}]\cite{tactics}. Unique feature for it's time include a isometric maps, a overworld map\footnote{A overworld map has a series of user selectable location, where the battle take place.} and  a branching storyline. It also popularised dynamic unit ordering that is determined by the character's attributes rather then each player moves all their units when its their turn.

	\item \emph{Final Fantasy Tactics}(1997) probably the most prolific TRPG in America/Europe. It featured a 3d viewpoint that imitated  the 2d isomeric view of previous games.  It allows the user to rotate the map which allowed less restricted maps\footnote{Older TRPG's maps could not be rotated hence they were designed such that tiles with greater height were at the far ends of the map. This was done to made sure that nothing was obscured.}.
	
	\item  \emph{Disgaea: Hour of Darkness}(2003): The first in a series of game that had the unique feature allow the player to play random generated maps to improve replayability. The latest in the series (``Disgaea 4'') is one of the few TRPGs that contain a map editor, which allow the player some limited customisable. 


	\item  \emph{Valkyria Chronicles}(2008) started the tend of blurring tactics with other genres. The game played like a typical TRPG \emph{except} when attacking an opponent where it acts like a first person shooter.

\end{itemize}

%TODO more refs

\subsection{Tactical RPG Game Engines}
\label{sub:tactical_rpg_game_engines}

Unlike other genres such as interactive fiction\cite{Bogdan2010Interactive-Fic} or even traditional RPGs\footnote{RPG have many engines available such as RPG Maker \cite{rpgmaker}} there are hardly any engine specific to Tactical RPG developers preferring to use their own engines or a general purpose engine such as Unity\footnote{Used to produce many independent games such Mysterious Castle and Grotesque Tactics: Evil Heroes}

One of the only complete\footnote{There are many half finished engine, which have been abandoned, mainly due to difficulty or lack of interest.} and successful TRPG creator is ``Simulation RPG Maker 95''\footnote{Simulation RPG being another name for a TRPG.}.  As apparent from its name it is extremely old but it does allow the creation of fairly complex TRPG, utilising a top down view. A notable of feature of the application is that it allowed visual map making and allowed a high degree of customisation to the units. 

The project's editor will be evaluated against the the described application in terms of features and easy of use. 

