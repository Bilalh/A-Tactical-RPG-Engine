%%Context Survey
%!TEX root = ../Project.tex

\section{Context Survey} 
\label{Context_Survey}

\subsection{Evolution of Tactical RPGs }
\label{sub:evolution_of_tactical_rpgs_}

Classic Tactical RPGs(TRPGs)  were heavy influenced by traditional Strategy games, in particular chess.  The earliest game that shares the concepts of a TRPG is ``Bokosuka Wars''(1983) which had many of the features now common in TRPGs including grid based combat with a number of units\cite{BokosukaWars}\footnote{Tactical RPG are also called Simulation Strategy RPG(SRPG),  most commonly in Japan.}. 

Although there were many western games that had similar gameplay at the time, like ``Ultima III: Exodus''(1985), which is a traditional RPG with TRPG like battles, they quickly diverged and soon featured real time combat, or reverted to a simple combat system like traditional RPG such as ``Final Fantasy''.  This tread led to most TRPG being made in Japan since, until recently, western publishers  thought TRPGs were too niche to be profitable \cite{notrpg}. 

I will briefly describe  the games that made a significant contribution to the genre, these will be used in evaluating the capabilities of the created engine.

\begin{itemize}
	
	\item \emph{Fire Emblem}(1990) popularised the genre. The game introduced RPG elements such as weapons and skills that became standard in later games. It also was one of the first TRPG to have a non-trivial plot. 

	\item \emph{Tactics Ogre}(1994) was the game which started the trend to have `tactics' in the name  which gave rise to the Tactical RPG name\footnotemark[\value{footnote}]\cite{tactics}. A unique feature for it's time included isometric maps, an overworld map\footnote{A overworld map has a series of user selectable locations, where the battle can take place.} and  a branching storyline. It also popularised dynamic unit ordering that is determined by the character's attributes rather than each player moving all their units when it's their turn.

	\item \emph{Final Fantasy Tactics}(1997) was probably the most prolific TRPG in America/Europe. It featured a 3D viewpoint that imitated  the 2D isomeric view of previous games.  It allowed the user to rotate the map which allowed less restricted maps\footnote{Older TRPGs' maps could not be rotated hence they were designed such that tiles with greater height were at the far ends of the map. This was done to make sure that nothing was obscured.}.
	
	\item  \emph{Disgaea: Hour of Darkness}(2003): The first in a series of games that had the unique feature allowing the player to play randomly generated maps to improve replayability. The latest in the series (``Disgaea 4'') is one of the few TRPGs that contains a map editor, which allow the player some limited customisability. 


	\item  \emph{Valkyria Chronicles}(2008) started the tend of blurring TRPG with other genres. The game played like a typical TRPG \emph{except} when attacking an opponent, where it acts like a first person shooter.

\end{itemize}

%TODO more refs

\subsection{Tactical RPG Game Engines}
\label{sub:tactical_rpg_game_engines}

Unlike other genres such as interactive fiction\cite{Bogdan2010Interactive-Fic} or even traditional RPGs\footnote{RPGs have many engines available such as RPG Maker \cite{rpgmaker}}, there are hardly any engines specific to Tactical RPG, developers preferring to use their own engines or a general purpose engine such as Unity\footnote{Used to produce many independent games such Mysterious Castle and Grotesque Tactics: Evil Heroes}

One of the only complete\footnote{There are many half finished engine, which have been abandoned, mainly due to difficulty or lack of interest.} and successful TRPG creator is ``Simulation RPG Maker 95''\footnote{Simulation RPG being another name for a TRPG.}.  As apparent from it's name, it is extremely old but it does allow the creation of fairly complex TRPG, utilising a top down view. Notable features of the application is that it allowed visual map making and it allowed a high degree of customisation to the units. 

The project's editor will be evaluated against the described applications in terms of features and ease of use. 

