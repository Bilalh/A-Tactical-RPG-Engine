%% Implementation
%!TEX root = ../Project.tex

\section{Implementation}

\subsection{Engine Development and Testing}
\label{sub:engine_development_and_testing}

\subsubsection{Maps}
\label{ssub:maps}

As discussed previously the maps use xml as their data format, one the advantages of this was that it required very little changes to the data  format to have incompatibility with Oleksandr Stasyk's  Terrain Generator's output format.  This allows the user design a map with very little work. 

\subsubsection{Units}
\label{ssub:units}

\subsubsection{Events}
\label{ssub:events}

\subsubsection{Algorithms}
\label{ssub:Algorithms}


\subsection{Gui Development and Testing}

\subsubsection{Map Rendering}
\label{ssub:map_rendering}

\subsection{User Interface}


\subsubsection{Custom Classes} % allows user to use their own code
\label{ssub:custom_classes}


\subsection{Editor Development and Testing}

\subsubsection{Overview}
\label{ssub:overview}

\subsubsection{Map Editor}
\label{ssub:map_editor}

\subsubsection{Unit Editor}
\label{ssub:unit_editors}

\subsubsection{Event Editing}

\subsubsection{Exporting}
\label{ssub:exporting}

The editor can export the game as a complete package, either as a Mac OS X application or as jar. These application don't require any external resources, apart from a recent version of java\footnote{specifically Java 1.6+}.

A prominent feature of the editor is that the jar will work on any Java enabled platform, since the jar contains all required libraries for each platform. The OS X application can even be exported on other platforms.

While most of the testing was done on OS X \footnote{Mac OS X 10.6 Snow leopard}, it also works well on Linux \footnote{Science  Linux x.y}. It even has limited compatibly with Windows\footnote{Tested on Windows 7 32 bit} (apart from some minor graphics issues).
