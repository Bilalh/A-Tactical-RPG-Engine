%% Objectives
%!TEX root = ../Project.tex
\subsection{Objectives}
\label{objectives}
% \begin{figure}[h!]
% 	\caption*{Key}
% 	\begin{center}
% 		\begin{tabular}{rl}
% 		\Checkmark & Finished\\
% 		\XSolidBrush & Not Started\\
% 		–  &  Partly\\
% 		\end{tabular}
% 	\end{center}
% \end{figure}

In the following subsubsections \Checkmark means that the objective is fully completed, \XSolidBrush means incompletion and \textbf{-} signifies partial  completion.


\subsubsection{Primary}
\label{primary}
The main goal of the primary objectives is to allow the user to create a complex Tactical RPG, with limited customisability.  
\begin{itemize}
\tick To develop an engine that takes:
\begin{itemize}

 	\tick The definition of character attributes and a combat system.
	\tick The definition of a world broken up into the smaller environments.
	\begin{itemize}
		\tick The rules of the game.
		\tick The kinds of enemies.
	\end{itemize}
	
	\tick The definition of a simple story as a wrapper for the whole game, from the start to the conclusion of the game
	\begin{itemize}
		\tick Which is told between the movement between different environments.
	\end{itemize}
	                        
\end{itemize}
and create a playable tactical RPG.

\tick To include in the engine support for the following:

\begin{itemize}
	\tick \texttt{units} with a fixed set of associated attributes such as:
	\begin{itemize}
		\tick Hit-points (which represent the health of the unit).
		\tick Strength.
		\tick Defence.
		\tick Move (The number of tiles the unit can move each turn).
	\end{itemize}
	
	\tick \texttt{battles} which take place on grid and include:
	\begin{itemize}
		\tick  A set number of \texttt{units} for each player.
		\tick  A Winning \texttt{condition}, which is to defeat all of the other player's units.
		\tick  Battles are \texttt{turn based} meaning only one unit performs at one time.   
		\tick  A combat system.
	\end{itemize}
	
	\tick A combat system that includes
		\begin{itemize}
			\tick \texttt{combat} between adjacent units.
			\begin{itemize}
				\tick When the unit hit-points are reduced to zero they are \texttt{defeated} and are removed from the map
			\tick A set of rules that govern the combat.
			\end{itemize}
			
		\end{itemize}
	
	\tick A predefined set of behaviours for how the non-player characters should behave.
	\begin{itemize}
		\tick Including pathfinding.
	\end{itemize}
	
	\tick An isometric graphical representation of the game.
	\begin{itemize}
		\tick Which shows the grid with all the units.
		\tick Allows the user to move their units and see the opponents moves.
		\tick Allows the user to attack the opponent's units.
		\tick Which allows the user to see a unit status (e.g current hit points).
		\tick Text will be used to describe the more complex actions such magic.
	\end{itemize}
\end{itemize}
\end{itemize}

\subsubsection{Secondary}
\label{secondary}

The main goal of the secondary objectives is to allow the user more customisability. 
\begin{itemize}
	\tick Tiles have \texttt{height}, where units can only move to tiles of a similiar height.
	\tick Tiles that are not passable such as sea, lava, etc \footnote{The engine as well as the GUI full supports impassable tiles. The editor has partial support only allows `empty' tiles }.
	
	\tick Tiles have different movement costs associated with them.
	
	\tick A combat system that includes 
	\begin{itemize}
		\tick \texttt{combat} between non-adjacent units,
	\end{itemize}
	
	\tick Players have items such as weapons that affect the result of combat between units. 
	\begin{itemize}
		\tick Including long distance weapons for the player and AI.
	\end{itemize}
	
	\partly Direction and height of the character's tile affects attack. \footnote{At the moment only the height affects the attack, while the direction is displayed in the GUI and changes based on the unit's movement, it not used in the model.}
	
	\tick Sound effects.
	\tick Music.
	
	\tick Saving and loading games.
	
	\tick Allow the user to specify some of behaviour of non-player characters
	\begin{itemize}
		\tick An example: always attack a certain kind of unit or always attack the unit with the least Hit Points.
	\end{itemize}
	
	\tick A graphical view to allow the user to specify input to the engine.
\end{itemize}

\subsubsection{Tertiary} 
\label{tertiary}
The goal of the Tertiary objectives are to provide the user with more customisability and to provide a GUI for customising aspects of the engine. 

\begin{itemize}
	
	\tick A combat system that includes 
	\begin{itemize}
		\tick Support for \texttt{skills} which can effect multiple units.
		\tick Including weapons that can attack multiple units at the same time. 
	\end{itemize}
	
	\tick Animations for units and movement.
	
	\tick A graphical editor for  creating and specifying the input to the engine which allows:
	\begin{itemize}
		\tick   Creating and editing maps.
		\begin{itemize}
			\tick which also allows placement of enemy units.
		\end{itemize}
		
		\tick   specifying the order of the maps.
		\tick   making animations.
		\tick   making items such as weapons.
		\tick   making skills. 
		\tick   making units.
		\tick   specifying the story, at the start and end of a battle.
		\tick   specifying the music and sound effects played on each map.
		\tick   specifying the condition to win a map such as:
		 \begin{itemize}
		 	\tick Defeating all the opponent's units.
		 	\tick Defeating a specific unit.
		 \end{itemize}
		\tick   specifying some of the behaviour of the enemy units.
		\tick   Allows exporting the game as a self contained application.
	\end{itemize}
	
	\cross Custom events
	\begin{itemize}
		\item Attached to units or titles, could be used for:
		\begin{itemize}
			\partly Making the player win if some enemies unit has less then 50\% Hit Points.
			\cross Damaging a character if step on a specified tile.
			\partly Showing some part of the story when a player's character reaches a specified tile.\footnote{The engine and GUI support displaying dialog at any time, but the editor only support creation of dialog for the start and end of a battle.}
		\end{itemize}
	\end{itemize}
	
\end{itemize}