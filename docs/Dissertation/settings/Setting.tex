%!TEX root = ../Project.tex
\providecommand{\setflag}{\newif \ifwhole \wholefalse}
\setflag
\ifwhole\else

\documentclass[11pt,a4paper]{article}
\def\studentID{080008164}
\def\fullName{Bilal Syed Hussain}
\newcommand{\theTitle}{}
\newcommand{\theAuthor}{\studentID}

\renewcommand{\theTitle}{A Tactical RPG Engine}
\renewcommand{\theAuthor}{\fullName}

\usepackage{multirow}
\usepackage{ifthen}
% \usepackage{UKenglish}                           
\bibliographystyle{IEEEtran}
\usepackage{color}
% \usepackage[bookmarks=true,pagebackref=true]{hyperref}
\usepackage[bookmarks=true]{hyperref}
\usepackage{pdfpages} %inculde pdfs
\hypersetup{
    pdftitle={\theTitle},    
    pdfauthor={Bilal Hussain},
    pdfsubject={Tactical RRG engine},   
    pdfcreator={Bilal Hussain},   
    pdfkeywords={Tactical RRG} {AI} {Game engine},, % list of keywords
    colorlinks=true,       % false: boxed links; true: colored links
    linkcolor=blue ,       % color of internal links
    citecolor=red,         % color of links to bibliography
    filecolor=red,         % color of file links
    urlcolor=red           % color of external links
}

\usepackage{setspace}
%\singlespacing
\onehalfspacing
%\doublespacing
%\setstretch{1.1}

% Listing (fold)
\usepackage{listings} 
\usepackage{textcomp} %for upquotes 
\usepackage{latexsym}
\usepackage{longtable}           
% to make it look nicer
\usepackage[T1]{fontenc}
\usepackage{tgcursor}
\lstset{
	tabsize=4,
	mathescape=true, 
	escapechar=§,
	upquote=true,
	basicstyle=\small\ttfamily,
	numberstyle=\tiny,
	numbersep=5pt,
	extendedchars=true,
	breaklines=true,
	keywords=[1]{weapon}
}

\lstnewenvironment{lst:weapon}[1][]
	{
		\lstset{
		keywords=[1]{weapon,name},
		keywords=[2]{class,name,uuid},
		#1	
		}
	}
	{}

\lstnewenvironment{lst:tile}[1][]
	{
		\lstset{
		keywords=[1]{tile,type,height},
		keywords=[2]{x,y},
		#1	
		}
	}
	{}

\lstnewenvironment{lst:resource}[1][]
	{
		\lstset{
		keywords=[1]{name,entry,uuid,resource},
		#1	
		}
	}
	{}

%\lstset{frame=b}
%\lstset{xleftmargin=17pt, framexleftmargin=17pt,framexrightmargin=5pt,framexbottommargin=4pt}
% for nicer captions
\usepackage{caption}
\DeclareCaptionFont{blue}{\color{blue}}
\DeclareCaptionFont{white}{\color{white}}
\DeclareCaptionFormat{listing}{\colorbox[cmyk]{0.43, 0.35, 0.35,0.01}{\parbox{\textwidth}{\hspace{15pt}#1#2#3}}}
\captionsetup[lstlisting]{format=listing,labelfont=white,textfont=white, singlelinecheck=false, margin=0pt, font={bf,footnotesize}}

\usepackage{sectsty}

%(end)

% comments
\newcommand{\Comment}[1]{}


% Pictures (fold)
\usepackage{graphicx}
\usepackage{subfigure}
\usepackage{wrapfig} % (end)
\usepackage{geometry}
% \geometry{bottom=2.4cm, top = 2.4cm}
\renewcommand{\topfraction}{0.97}	% max fraction of floats at top
\renewcommand{\bottomfraction}{0.97}	% max fraction of floats at bottom
\renewcommand{\textfraction}{0.01}

% Very small margins
% \geometry{left=1.5cm, right=1.5cm, bottom=1.9cm, top = 1.9cm}
% (end)

% Maths (fold)
\usepackage{amsmath}
\usepackage{amssymb}
\usepackage{amsthm}

% Makes small lists
\usepackage{enumitem} 
\usepackage[utf8]{inputenc}


% Not on Packages/Settings (fold)
%small sections
 % \usepackage[small,compact]{titlesec}
% Small text nicer
 % \usepackage{microtype}

% \usepackage{alltt}
% \renewcommand{\ttdefault}{txtt}

% \usepackage[bookmarks=true]{hyperref}
% \setcounter{secnumdepth}{5}
% \setcounter{tocdepth}{5}

% (end)


% shortcuts (fold)

\newcommand{\picwidth}{6.3in}

\def\*{\ensuremath{\times}}
\def\<={\ensuremath{\leqslant}}
\def\>={\ensuremath{\geqslant}}
\def\={\ensuremath{\neq}}
\def\~{\ensuremath{\approx}}

\def\s{\textsuperscript{* }}
\def\ts{\textsuperscript}
\def\-{\subscript}
\def\_{\textunderscore}
\def\set#1{\{ #1 \}}
\def\t#1{\text{#1}}

\def\abs#1{\mathopen| #1 \mathclose|}
\def\Integer{\mathsf{Z\hspace{-0.4em}Z}}
\def\Natural{\mathrm{I\!N}}
\def\Real{\mathrm{I\!R}}
%(end)

% Headers footers (fold)             
\headheight14pt
%	default Headers
% \pagestyle{headings}

\usepackage{fancyhdr} 
\pagestyle{fancy}
\renewcommand{\headrulewidth}{0pt} % remove lines as well
\fancyhf{} % removes all headers


\newcommand{\currentSection}{
	% Inculdes subsections
	% \ifthenelse{\equal{\rightmark}{}}{\leftmark}{\rightmark}
	% Only topmost sections
	\leftmark	
}

\ifthenelse{\boolean{@twoside}{false}}
{
	\let\tmp\oddsidemargin
	\let\oddsidemargin\evensidemargin
	\let\evensidemargin\tmp
	\reversemarginpar
	\fancyhead[LO,RE]{\slshape \nouppercase{\currentSection}} 
	\fancyhead[LE,RO]{\thepage} 
	\fancyhead[CE]{\hspace{-30pt} \theAuthor}
}{
	\fancyhead[LO,RE]{\slshape \nouppercase{\currentSection}} 
	\fancyhead[LE,RO]{\thepage} 
	\fancyhead[CO]{\hspace{30pt}  \theAuthor  \hspace{5pt}}
	\fancyhead[CE]{\hspace{-50pt} \theAuthor  \hspace{5pt}}
}

% (end)      


% for lists
\usepackage{bbding}
\newcommand*\tick{\item[\Checkmark]}
\newcommand*\cross{\item[\XSolidBrush]}
% \newcommand*\partly{\item[\ensuremath{\bullet}]}
\newcommand*\partly{\item[–]}

% question macros (fold) 
\newcounter{qnumber}
\setcounter{qnumber}{0}
\stepcounter{qnumber}
  
\def\qquestion#1\par#2\par{\hbox to \hsize
{\vbox{\hsize=0.5\hsize #1}\quad#2\hfil}\medskip\goodbreak}

\def\freequestion#1\par{#1\par\nobreak
	\begingroup\nobreak
	\advance\leftskip by 2pc
	\hrule width 0pt height 1.7\baselineskip\hrulefill
	\par
	\medskip
	\endgroup
}

\def\boxit#1{\hbox{\lower0.7ex\vbox{\hrule\hbox{\vrule\kern1pt
	\vbox{\kern1pt\hbox to 1.4em
	{\small\strut\hfil #1\hfil}\kern1pt}\kern1pt\vrule}\hrule}}
}

\def\fiveboxes#1#2#3#4#5{\hbox to\scalewidth
	{
		\boxit{#1}\hfil\boxit{#2}\hfil\boxit{#3}\hfil
		\boxit{#4}\hfil\boxit{#5}
	}
}

	\def\xagree{\xscale{strongly disagree}{agree completely}}
	\def\boxes{\fiveboxes{}{}{}{}{}\ignorespaces}
	\def\xscale#1#2{%
		 \setbox0=\hbox{\boxes}%
		 \setbox2=\hbox to \wd0{\small\strut $\gets$ #1 \hfill \small\strut\hfill #2 $\to$}
		 \vbox{\vbox to 0pt{\vss\box1\box2\kern2pt}\vbox{\box0}}
	}
	\def\xboxes{
		\setbox0=\hbox{\boxes}
		\vbox{\vbox to 0pt{\vss\box1\box2\kern2pt}\vbox{\box0}}
	}
	\newdimen\scalewidth
	\scalewidth=0.5\hsize

\newcommand{\questionNumber}{\theqnumber.\;\stepcounter{qnumber}}
\newcommand{\startingQuestion}[1]{
	\setcounter{qnumber}{1} 
	\qquestion\noindent \questionNumber #1
	\par\hspace{15pt}\xagree \vspace{5pt}
}
\newcommand{\question}[1]{
	\qquestion\noindent \questionNumber #1 
	\par\hspace{10pt} \xboxes \vspace{5pt} 
}

\newcommand{\fquestion}[1]{
	\freequestion\noindent \questionNumber \noindent #1
	\par
}
%Question macors (end)

% Foreach  (fold)
\usepackage{etoolbox}
\makeatletter

% Functional foreach construct 
% #1 - Function to call on each comma-separated item in #3
% #2 - Parameter to pass to function in #1 as first parameter
% #3 - Comma-separated list of items to pass as second parameter to function #1
\def\foreach#1#2#3{%
  \@test@foreach{#1}{#2}#3,\@end@token%
}

% Internal helper function - Eats one input
\def\@swallow#1{}

% Internal helper function - Checks the next character after #1 and #2 and 
% continues loop iteration if \@end@token is not found 
\def\@test@foreach#1#2{%
  \@ifnextchar\@end@token%
    {\@swallow}%
    {\@foreach{#1}{#2}}%
}

% Internal helper function - Calls #1{#2}{#3} and recurses
% The magic of splitting the third parameter occurs in the pattern matching of the \def
\def\@foreach#1#2#3,#4\@end@token{%
  #1{#2}{#3}%
  \@test@foreach{#1}{#2}#4\@end@token%
}

\makeatother
% (end)

\fi
