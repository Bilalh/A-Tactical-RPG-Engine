%!TEX root = ../Project.tex
\section{Scripting}
\label{sec:Scripting}

Scripting allows the user to customise aspects of the game. This includes customising the opponent's AI, custom winning conditions and user defined events. 

\subsection{Language Choice}

There were three main choices using Javascript, using JRuby\footnote{A Ruby implementation written in java}, or building a `domain specific language'.

Creating a `domain specific language' was considered initially, this would have the following advantages:
\begin{itemize}
	\item Provides more abstraction, and allow the complex details to be hidden.
	\item Easier to validate since the languages contains a very few constructs.  
\end{itemize}

\noindent but was rejected because:

\begin{itemize}
	\item of the time to create and test the new language.
	\item of the cost of creating tools for the new language, there are already source code highlighters and debuggers for Javascript and Ruby.
	\item of the loss of efficiency, the Javascript parser in the JDK as well as JRuby is very efficient and provides advance features such as `just in time compilation' 
	\footnote{A method to improve the runtime performance, by translating the interpreted code into lower level form, while the code is be run
	}  
which would not be possible to implement for the new language within the time constraint of the project.
\end{itemize}

\noindent JRuby has the following advantage:
\begin{itemize}
	\item Easier syntax for interacting with Java then javascript.
	\item Easy to use with the embedding API in the JDK.
\end{itemize}

\noindent Javascript was chosen over Ruby as a scripting language for the following reasons:
\begin{itemize}
	\item Javascript embedding is build into the JDK, so the user does not have install anything extra. It also has the advantage of being cross platform. 
	\item Javascript is easy to learn, and average user is more likely to have used it before as compared to Ruby.
\end{itemize}

\subsection{Data Exposed}
	Events can be attached to \texttt{units}, \texttt{tiles} in a battle, globally in a battle and to the AI.
All events are passed a \texttt{mapinfo} object which contains the following as read only data:
\begin{itemize}
	\item A hashtable of the players unit and a hashtable of the enemies units. For each unit this includes 
		\begin{itemize}
			\item all the unit's attributes such as the location, and \texttt{hit points}. 
			\item if the unit has been defeated. 
		\end{itemize}
	\item The \texttt{leader} unit of each side if there is one.
	\item The number of turns taken.
\end{itemize}
The \texttt{mapinfo}  object contains the following methods:
\begin{itemize}
	\item[\texttt{win}]     The player wins the battle.
	\item[\texttt{lose}]    The player loses the battle.
	\item[\texttt{dialog}]  The player is shown the specified dialog (to show the user some the plot). Can be directed from a specify unit, or a global message.
	\item[\texttt{action}]   Executes the specified action.
\end{itemize}

This allows the user to make complex events without them changing the model to much.

\subsection{Action}
A \texttt{action} is a set of unit defined actions. For example a poison action could reduces the a units `hit points' by 10\% 

\subsection{Winning Conditions}
The user can specify the winning conditions based on what occurring in the battle, examples include
\begin{itemize}
	\item If opponent's leader's hp $< 50\%$ then \texttt{win()}.
	\item If <character> dies then lose().
	\item If number of turns > 20 then lose()
\end{itemize} 

\subsection{Unit Events}

Unit events get passed the specified unit as well as the \texttt{mapinfo}. the event can be specified to execute when:
\begin{enumerate}
	\item The unit finishes its turn.
	\item The unit is affected by magic.
	\item The unit is attacked.
	\item The unit attacks.
\end{enumerate}
Example: When <unit> attacked counter attack.


\subsection{Tiles Events}

Tiles get passes the specified tile as well as the Unit. The event can be specified to execute when
\begin{itemize}
	\item A unit moves to a tile.
	\item A unit moves though a tile.
\end{itemize}

Example: On unit moving though \texttt{action(posion)}

\subsection{AI Events}

The behaviour of AI can be customised, with \texttt{commands} such as:
\begin{itemize}
	\item Attack the player's unit with  highest/lowest hp.
	\item Attack the player's leader unit.
	\item If player's leader's hp $< 20\%$ \texttt{heal(leader)}.
	\item Attack player's characters of class <class>.
\end{itemize}

The AI events \texttt{mapinfo} has addition methods including:
\begin{itemize}
	\item[\texttt{attack}]  Attack the specified unit.
	\item[\texttt{follow}]  Move as close as possible to the specified unit.
	\item[\texttt{heal}]    Heal the specified unit.
	\item[\texttt{move}]    Move to the specified location.
	\item[\texttt{wait}]    Do nothing this turn
\end{itemize}   

% \clearpage
The \texttt{commands} themselves can be conditional, as example 
\begin{lstlisting}[caption=Conditional AI Event ]
If opponent's leader's hp $< 20\%$ then
	heal(leader).
else If player has a leader unit then
	If player's leader's hp $< 20\%$  then
	 	Attack the player's leader unit
	else
		Attack the player's closet unit with the lowest hp.
	end
else
	wait
end	
\end{lstlisting}

