%!TEX root = ../Project.tex
\section{Scripting}
\label{sec:Scripting}

Scripting allows the user to customise aspects of the game. This includes customising the opponent's AI, custom winning conditions and user defined events. 

\subsection{Language Choice}

There were three main choices using Javascript, using JRuby\footnote{A Ruby implementation written in java}, or building a `domain specific language'.

Creating a `domain specific language' was considered initially, this would have the following advantages:
\begin{itemize}
	\item Provides more abstraction, and allow the complex details to be hidden.
	\item Easier to validate since the languages contains a very few constructs.  
\end{itemize}

\noindent but was rejected because:

\begin{itemize}
	\item of the time to create and test the new language.
	\item of the cost of create tools for the new language, there are already source code highlighter and debugger for java.
	\item of the loss of efficiency, the Javascript parser in the JDK as well as JRuby is very efficient and provides advance features such as `just in time compilation' 
	\footnote{A method to improve the runtime performance, by translating the interpreted code into lower level form, while the code is be run
	}  
which would not be possible to implement for the new language within the time constraint of the project.
\end{itemize}

\noindent JRuby has the following advantage:
\begin{itemize}
	\item Easier syntax for interacting with Java then javascript.
	\item Easy to use with the embedding API in the JDK.
\end{itemize}

\noindent Javascript was chosen over Ruby as a scripting language for the following reasons:
\begin{itemize}
	\item Javascript embedding is build into the JDK, so the user does not have install anything extra. It also has the advantage of being cross platform  
	\item Javascript is easy to learn, and average user is more likely to have used it before as compared to Ruby.
\end{itemize}



\subsection{General}

\begin{itemize}

\item Attached events to units, titles etc.

\item Events get passed a \texttt{mapinfo} object containing.
\begin{itemize}
	\item a hashtable of the players unit and a hashtable of the enemies units..
	\item The leader of each side if there is one.
	\item The number of turns taken.
\end{itemize}

\item \texttt{mapinfo} also contains methods such as \texttt{Win} and \texttt{Lose} to allow custom victory conditions.

\item \texttt{dialog} method to make unit talk.
\begin{itemize}
	\item Either on unit or not.
\end{itemize}

\item Unit events get passed the specified unit. Executed when:
\begin{enumerate}
	\item the units finishes it turn.
	\item is affected by magic.
	\item attacked.
\end{enumerate}

\item Tiles get passes the specified tile. Executed when:
\begin{itemize}
	\item A unit moves onto the tile.
\end{itemize}

\end{itemize}

\subsection{Rules}
\subsubsection{AI}
\begin{itemize}
	\item Attack the player's unit with  highest/lowest hp.
	\item Attack the player's leader unit (if there is one).
	\item If opponent's leader's hp $< 20\%$ \texttt{heal(leader)}.
	\item Attack player's characters of class <class>.
\end{itemize}

\subsubsection{Events}
\paragraph{Character}
\begin{itemize}
	\item If opponent's leader's hp $< 50\%$ then \texttt{win()}.
	\item If <character> dies then \texttt{lose()}.
\end{itemize}

\paragraph{World}
\begin{itemize}
	\item If \texttt{number\_of\_turns} $> 20$ then \texttt{lose()} 
\end{itemize}

\paragraph{Tile}
\begin{itemize}
	\item If <character> enters then \texttt{event(id)} 
\end{itemize}

